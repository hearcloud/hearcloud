\documentclass[a4paper,11pt]{book}
\usepackage{listings}
\usepackage[utf8]{inputenc}
\usepackage[spanish, es-tabla]{babel}

\usepackage{titlesec}
\usepackage{fancyhdr}
\usepackage[hidelinks]{hyperref}
\usepackage{xcolor}
\usepackage{pdfpages}

\usepackage{xspace}

% ********************************************************************
% Información reutilizable
% ********************************************************************
\newcommand{\mySubject}{Trabajo Fin de Grado\xspace}
\newcommand{\myTitle}{Hearcloud\xspace}
\newcommand{\mySubTitle}{Aplicación web orientada a la organización y gestión de música online\xspace}
\newcommand{\mySubTitleENG}{Web aplication for music management\xspace}
\newcommand{\myDegree}{Grado en Ingeniería Informática\xspace}
\newcommand{\myName}{Mariano Palomo Villafranca\xspace}
\newcommand{\myEmail}{mpvillafranca@correo.ugr.es\xspace}
\newcommand{\myDNI}{20100242J\xspace}
\newcommand{\myProf}{José María Guirao Miras\xspace}
\newcommand{\myFaculty}{Escuela Técnica Superior de Ingenierías Informática y de
Telecomunicación\xspace}
\newcommand{\myFacultyShort}{E.T.S. de Ingenierías Informática y de
Telecomunicación\xspace}
\newcommand{\myDepartment}{Departamento de Lenguajes y Sistemas Informáticos\xspace}
\newcommand{\myUni}{\protect{Universidad de Granada}\xspace}
\newcommand{\myLocation}{Granada\xspace}
\newcommand{\myTime}{\today\xspace}
\newcommand{\myVersion}{Version 0.1\xspace}

% ********************************************************************
% Información de archivo
% ********************************************************************
\hypersetup{
pdfauthor = {\myName (\myEmail)},
pdftitle = {\myTitle},
pdfsubject = {\mySubject},
pdfkeywords = {Hearcloud, aplicación web, música, Django, software libre},
pdfcreator = {LaTeX con el paquete Texmaker},
pdfproducer = {pdflatex}
}

% ********************************************************************
% Estilo de cabeceras
% ********************************************************************
\pagestyle{fancy}
\fancyhf{}
\fancyhead[LO]{\leftmark}
\fancyhead[RE]{\rightmark}
\fancyhead[RO,LE]{\textbf{\thepage}}
\setlength{\headheight}{1.5\headheight}

% ********************************************************************
% Redefinición de comandos
% ********************************************************************
\renewcommand{\chaptermark}[1]{\markboth{\textbf{#1}}{}}
\renewcommand{\sectionmark}[1]{\markright{\textbf{\thesection. #1}}}

% ********************************************************************
% Definición de colores
% ********************************************************************
\definecolor{gray97}{gray}{.97}
\definecolor{gray75}{gray}{.75}
\definecolor{gray45}{gray}{.45}
\definecolor{gray30}{gray}{.94}

% ********************************************************************
% Listados
% ********************************************************************
\lstset{ frame=Ltb,
     framerule=0.5pt,
     aboveskip=0.5cm,
     framextopmargin=3pt,
     framexbottommargin=3pt,
     framexleftmargin=0.1cm,
     framesep=0pt,
     rulesep=.4pt,
     backgroundcolor=\color{gray97},
     rulesepcolor=\color{black},
     %
     stringstyle=\ttfamily,
     showstringspaces = false,
     basicstyle=\scriptsize\ttfamily,
     commentstyle=\color{gray45},
     keywordstyle=\bfseries,
     %
     numbers=left,
     numbersep=6pt,
     numberstyle=\tiny,
     numberfirstline = false,
     breaklines=true,
   }
 
% Minimizar fragmentado de listados
\lstnewenvironment{listing}[1][]
   {\lstset{#1}\pagebreak[0]}{\pagebreak[0]}

% ********************************************************************
% Estilo de listados de código
% ********************************************************************
\lstdefinestyle{CodigoC}
   {
	basicstyle=\scriptsize,
	frame=single,
	language=C,
	numbers=left
   }
\lstdefinestyle{CodigoC++}
   {
	basicstyle=\small,
	frame=single,
	backgroundcolor=\color{gray30},
	language=C++,
	numbers=left
   }

 
\lstdefinestyle{Consola}
   {basicstyle=\scriptsize\bf\ttfamily,
    backgroundcolor=\color{gray30},
    frame=single,
    numbers=none
   }


\newcommand{\bigrule}{\titlerule[0.5mm]}

% ********************************************************************
% Para que en las páginas en blanco no tengan cabeceras
% ********************************************************************
\makeatletter
\def\clearpage{%
  \ifvmode
    \ifnum \@dbltopnum =\m@ne
      \ifdim \pagetotal <\topskip
        \hbox{}
      \fi
    \fi
  \fi
  \newpage
  \thispagestyle{empty}
  \write\m@ne{}
  \vbox{}
  \penalty -\@Mi
}
\makeatother

% ********************************************************************
% Memoria
% ********************************************************************
\begin{document}
\begin{titlepage}
 
\newlength{\centeroffset}
\setlength{\centeroffset}{-0.5\oddsidemargin}
\addtolength{\centeroffset}{0.5\evensidemargin}

\noindent\hspace*{\centeroffset}\begin{minipage}{\textwidth}

\centering
\includegraphics[width=0.9\textwidth]{../images/logo_ugr.jpg}\\[1.4cm]

\textsc{\Large\mySubject\\[0.2cm]}
\textsc{\myDegree}\\[1cm]

{\Huge\bfseries \myTitle\\}
\noindent\rule[-1ex]{\textwidth}{3pt}\\[3.5ex]
{\large\bfseries \mySubtitle}
\end{minipage}

\vspace{2.5cm}
\noindent\hspace*{\centeroffset}\begin{minipage}{\textwidth}
\centering

\textbf{Autor}\\ {\myName}\\[2.5ex]
\textbf{Tutor}\\ {Prof. Dr. \myProf}\\[2cm]
\includegraphics[width=0.3\textwidth]{../images/etsiit_logo.png}\\[0.1cm]
\textsc{\myFaculty}\\
\textsc{---}\\
\myLocation, \myTime\\
\includegraphics[width=0.3\textwidth]{../images/CC-SA-logo.png}
\end{minipage}
\end{titlepage}



\begin{titlepage}
 
 
\setlength{\centeroffset}{-0.5\oddsidemargin}
\addtolength{\centeroffset}{0.5\evensidemargin}
\thispagestyle{empty}

\noindent\hspace*{\centeroffset}\begin{minipage}{\textwidth}

\centering

\vspace{3.3cm}

% Logo del proyecto
\includegraphics[width=0.5\textwidth]{../images/logo_hearcloud_full.png}

 \vspace{0.5cm}

% Title

{\Huge\bfseries \myTitle\\
}
\noindent\rule[-1ex]{\textwidth}{3pt}\\[3.5ex]
{\large\bfseries \mySubTitle.\\[4cm]}
\end{minipage}

\vspace{2.5cm}
\noindent\hspace*{\centeroffset}\begin{minipage}{\textwidth}
\centering

\textbf{Autor}\\ {\myName}\\[2.5ex]
\textbf{Tutor}\\ {Prof. Dr. \myProf}\\[2cm]

\end{minipage}
\vspace{\stretch{2}}

 
\end{titlepage}




\begin{center}
{\large\bfseries \myTitle. \mySubTitle}\\
\end{center}
\begin{center}
\myName\\
\end{center}

% ********************************************************************
% Resumen (Español)
% ********************************************************************
\section*{Resumen}

\bigskip
\noindent{\textbf{Palabras clave}: \textit{aplicación web}, \textit{cloud}, \textit{música}, \textit{organización}, \textit{servidor}, \textit{cliente} }\\

Se pretende desarrollar una \textit{aplicación web} que permita el almacenamiento de ficheros de \textit{música} teniéndolos disponibles en cualquier momento y lugar que se desee, facilitando su accesibilidad a modo de plataforma de computación en la nube (\textit{cloud}). Como resultado, tendremos una herramienta útil y sencilla orientada a la organización de la biblioteca musical del usuario, que podrá realizar modificaciones sobre los metadatos de sus ficheros, tales como ``título'', ``artista'' o ``portada'' y descargarlos de nuevo al sistema local, si quiere, tras los cambios realizados.

El proyecto estará basado en su totalidad en el \textit{software libre}, pudiendo ser adaptado sin dificultad por cualquier otro usuario que así lo quisiera. El desarrollo del servidor se realizará con {\tt Django}, que es un framework para aplicaciones web gratuito y open source, escrito en {\tt Python} que ayuda a desarrollar sitios web más fácil y rápidamente. Además, haremos uso de {\tt Jinja2} para generar archivos \textit{HTML} basados en \textit{templates}. Por otra parte, en el lado del cliente, utilizaremos código \textit{CSS} para definir y crear la presentación de los documentos HTML con el framework {\tt Twitter Bootstrap} que contiene plantillas de diseño para tipografía, formularios, botones, cuadros, menús de navegación y otros elementos de diseño y nos simplificará la tarea. Por otra parte, también se incluirá código \textit{Javascript} y \textit{jQuery} que ayudará a mejorar la experiencia de usuario durante su navegación por el sistema.

Al tratarse de \textit{software libre}, el desarrollo no está limitado al equipo que inicia el trabajo y cualquier usuario puede hacer una contribución al mismo. Para ello, se utilizará un \textit{sistema de control de versiones}, en este caso, {\tt Git}, que es prácticamente un estándar en el mundo. Además, se empleará la plataforma de desarrollo colaborativo (forja) {\tt Github}, que está integrada con Git y nos ayudará en el desarrollo y la distribución del trabajo. De esta forma, cualquier persona que quiera acceder al código del proyecto, podrá hacerlo a través del repositorio y obtener su propia copia gratuita.




\newpage

% ********************************************************************
% Resumen (Inglés)
% ********************************************************************
\begin{center}
{\large\bfseries \myTitle. \mySubTitleENG}\\
\end{center}
\begin{center}
\myName\\
\end{center}

\section*{Abstract}

\bigskip
\noindent{\textbf{Keywords}: Keyword1, Keyword2, Keyword3, ....}\\

Write here the abstract in English.

\newpage

% ********************************************************************
% Autorización de publicación en la biblioteca del centro
% ********************************************************************
%\chapter*{}
\thispagestyle{empty}
\noindent\rule[-1ex]{\textwidth}{2pt}\\[4.5ex]

Yo, \textbf{\myName}, alumno de la titulación \myDegree{} de la \textbf{\myFaculty}, con DNI \myDNI, autorizo la
ubicación de la siguiente copia de mi Trabajo Fin de Grado en la biblioteca del centro para que pueda ser
consultada por las personas que lo deseen.

\bigskip
Así mismo, dicha copia se publica bajo licencia \textbf{Creative Commons Attribution-ShareAlike 4.0}. Los terminos de la licencia permiten su copia y/o adaptación en cualquier medio o formato, siempre y cuando se reconozca la autoría y se redistribuya con la misma licencia que el original. La presente documentación, en formato {\tt LaTeX}, puede encontrarse en el repositorio de {\tt Github}: \url{https://github.com/mpvillafranca/hearcloud}.

\vspace{6cm}

\noindent Fdo: \myName

\vspace{2cm}

\begin{flushright}
\myLocation a \myTime.
\end{flushright}

\newpage


% ********************************************************************
% Tutor
% ********************************************************************
%\chapter*{}
\thispagestyle{empty}
\noindent\rule[-1ex]{\textwidth}{2pt}\\[4.5ex]

D. \textbf{\myProf}, profesor del \myDepartment de la \myUni.

\vspace{0.5cm}

\textbf{Informan:}

\vspace{0.5cm}

Que el presente trabajo, titulado \textit{\textbf{\myTitle, \mySubTitle}}, ha sido realizado bajo su supervisión por \textbf{\myName}, y autoriza la defensa de dicho trabajo ante el tribunal que corresponda.

\vspace{0.5cm}

Y para que conste, expiden y firman el presente informe en \myLocation a \myTime.

\vspace{1cm}

\textbf{El tutor:}

\vspace{5cm}

\noindent \textbf{\myProf}

\chapter*{Agradecimientos}
\thispagestyle{empty}

       \vspace{1cm}


Poner aquí agradecimientos...



\tableofcontents
\listoffigures
\listoftables

\mainmatter
\setlength{\parskip}{5pt}

\input{chapters/01_Introduccion}

\input{chapters/02_Objetivos}

\input{chapters/03_Analisis}

\chapter{Planificación}

\section{Fases y entregas}

\subsection{Fases}

\subsection{Lista de entregas}

\section{Estructura de descomposición del trabajo}

\section{Lista de actividades}

\section{Recursos humanos}

Se contempla la participación de representantes del público objetivo destinatario del producto con el fin de recabar información relativa a su utilidad, así como posibles deficiencias, beneficios y/o mejoras.

\section{Presupuesto}

La principal ventaja del uso de software libre es que no es necesario adquirir licencias de pago para usar dicho software. Como las herramientas que se van a utilizar y todo el código que se genere será libre, el coste económico y, por tanto, el presupuesto inicial del proyecto será cero.

\bigskip

El único aspecto económico relevante que habrá que considerar en este caso será, entonces, el de los posibles gastos por el uso de un servidor en el que se implate el software.

\section{Temporización}


\input{chapters/05_Diseno}

\input{chapters/06_Implementacion}

\input{chapters/07_Pruebas}

\input{chapters/08_Conclusiones}


%%\nocite{*}
%\bibliography{bibliografia/bibliografia}\addcontentsline{toc}{chapter}{Bibliografía}
%\bibliographystyle{miunsrturl}
%
%\appendix
%\input{apendices/manual_usuario/manual_usuario}
%%\input{apendices/paper/paper}
%\input{glosario/entradas_glosario}
% \addcontentsline{toc}{chapter}{Glosario}
% \printglossary
\chapter*{}
\thispagestyle{empty}
\end{document}
